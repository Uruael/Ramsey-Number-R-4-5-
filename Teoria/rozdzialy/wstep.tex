\chapter{Wstęp}

Nasza praca opiera sie na publikacji R(4,5) = 25 wydanej w 1995 przez Brendana D. McKaya oraz Stanisława P. Radziszowskiego  \cite{mainpaper}. Motywacją do sporządzenia tej pracy było udowodnienie, z wykorzystaniem nowych technologii, że dokładna wartość liczby Ramseya R(4,5) wynosi 25.  Motywacją do sporządzenia tej pracy było udowodnienie, że dokładna wartość liczby Ramseya R(4,5) wynosi 25, z wykorzystaniem nowych technologii. Prace nad wyznaczeniem wartości R(4,5) zaczęły się w 1955 wraz z wydaniem przez Greenwooda oraz Gleasona artykułu w którym wyznaczyli oni górną granicę $R(4,5) \leq 31$. W kolejnych latach granica ta była zawężana aż do 25 $\leq$ R(4,5) $\leq$ 27.\par
Jak pokazano wcześniej, wygenerowanie wszystkich możliwych dwukolorowych grafów a następnie sprawdzenie ich poprawności byłoby zbyt czasochłonne więc wymagane było inne podejście do problemu. Wykorzystano jedynie wyselekcjonowane grafy ($s$,$t$,$n$) gdzie $s$ oznacza rozmiar maksymalnej kliki która znajduje się w grafie, $t$ oznacza wielkość maksymalnego zbioru niezależnego który należy do grafu, a $n$ oznacza liczbę wierzchołków na których zbudowany jest graf. Celem było skonstruowanie rodziny grafów R(4,5,24) z grafów R(3,5,d) oraz R(4,4,24-d) gdzie 7 $\leq$ d $\leq$ 13, przy pomocy algorytmu nazwanym ''sklejaniem''. Ostatnim krokiem przed weryfikacją wyników jest próba rozszerzenia otrzymanych grafów R(4,5,24) o jeden wierzchołek.

W naszej pracy omawiamy techniki i algorytmy służące do przeprowadzenia powyższego dowodu, oraz opisujemy dokonaną implementację.

