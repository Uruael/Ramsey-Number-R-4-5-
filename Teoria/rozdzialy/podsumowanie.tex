\chapter{Podsumowanie}
Udało się utworzyć wszystkie komponenty potrzebne do pełnego działania programu, ale ze względu na czas wykonywania, pełen dowód nie został przeprowadzony. Zostały wygenerowane wymagane grafy R(3,5,n) oraz R(4,4,n) (algorytm 2), których sposób generacji został opisany w rozdziale 4. Stworzyliśmy i zaimplementowaliśmy algorytm, który na podstawie wcześniej wygenerowanych grafów, tworzy przedziały stożków prawdopodobnych, opisanych w podrozdziale 5.3. Sklejanie grafów, opisane w podrozdziale 5.3 oraz 5.4, także zostało wykonane i przetestowane pod kątem poprawności. Z powodu problemów z algorytmem sklejania, algorytm rozszerzania grafu o wierzchołek nie został wykorzystany w pełni do zamierzonych celów. Jego implementacja została użyta do podjęcia próby rozszerzenia wyników sklejana, ale za względu na niekompletność danych nie został przeprowadzony pełny dowód. Nie dokonaliśmy również sprawdzenia wydajności obliczeniowej naszej implementacji rozszerzania na dużych zbiorach grafów.

Cel który nie został zrealizowany to optymalizacja programu do tego stopnia aby wynik można było uzyskać w czasie realistycznie pozwalającym na wykonanie na niespecjalistycznej jednostce obliczeniowej. 

W przypadku chęci kontynuacji prac nad projektem należało by przeprowadzić optymalizacje tworzenia przedziałów stożków prawdopodobnych oraz samego sklejania grafów. Niestety nie da się tego zrobić korzystając jedynie z prostych heurystyk. Wszystkie usprawnienia mogące przyśpieszyć wykonanie programu wymagają implementacji złożonych algorytmów, które nie zostały w pełni przedstawione w pracy na której bazowaliśmy. 