\chapter*{Streszczenie}
\indent 

Wyznaczanie liczb Ramseya odnosi się do gałęzi matematyki która dotyczy teorii grafów. Liczba Ramseya R(4,5) jest określona jako najmniejsza liczba całkowita $n$ dla której $n$-wierzchołkowy graf zawiera albo klikę stopnia 4 lub zbiór niezależny stopnia 5. Praca opiera się na wynikach przedstawionych przez Brendana D. McKaya oraz Stanisława P. Radziszewskiego. Celem pracy jest odtworzenie eksperymentu potwierdzającego że R(4,5)=25, dzięki wykorzystaniu współczesnej technologi oraz przystosowanych do tego algorytmów. Praca przedstawia w jaki sposób można stworzyć program dokonujący dowodu, jednak nie wykorzystując wszystkich możliwych optymalizacji, co przekłada się na wysoki czas działania.
\vspace{0.5cm}\newline
\textbf{Słowa kluczowe:} graf Ramseyowski, liczba Ramseya, generowanie grafów
\vspace{0.5cm}

\noindent \textbf{Dziedzina nauki i techniki, zgodnie z wymogami OECD:} nauki inżynieryjne i techniczne, sprzęt komputerowy i architektura komputerów
