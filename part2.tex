\documentclass[11pt]{article}
 
\usepackage[T1]{fontenc}
\usepackage[polish]{babel}
\usepackage[utf8]{inputenc}
\usepackage{lmodern}
\usepackage{mathtools}
\selectlanguage{polish}
\usepackage{graphicx}
\usepackage{tikz} 
\newtheorem{theorem}{Twierdzenie} 
\newtheorem{definition}{Definicja}[section]
\graphicspath{ {./im/} }




\begin{document}

\section{Podstawy teoretyczne}

  Przed omówieniem tematu naszej pracy, należy przedstawić kilka pojęć z teorii grafów, bez których zrozumienia, nie jest możliwe wyznaczanie liczb Ramseya.

  \subsection{Izomorfizm}
  Izomorfizm jest to takie przekształcenie grafu, które zachowuje wszystkie jego właściwości. Innymi słowy dwa grafy są izomorficzne, jeżeli ich wierzchołki można nazwać w taki sposób, aby sąsiadami odpowiadających sobie wierzchołków w obu grafach miały takie same zbiory sąsiadów. 

  \begin{tikzpicture}[node distance={15mm}, main/.style = {draw, circle}] 
    \node[main] (1) {$1$};
    \node[main] (2) [below right of=1] {$5$};
    \node[main] (3) [above right of=2] {$2$};
    \node[main] (4) [below left of=2] {$4$};
    \node[main] (5) [below right of=2] {$3$};

    \draw (1) -- (2);
    \draw (2) -- (3);
    \draw (2) -- (4);
    \draw (2) -- (5);
    \draw (1) -- (3);
    \draw (4) -- (5);
  \end{tikzpicture}

  Przekształćmy ten graf zamieniając wierzchołek 1 z 2 oraz 3 z 4:

  \begin{tikzpicture}[node distance={15mm}, main/.style = {draw, circle}] 
    \node[main] (1) {$2$};
    \node[main] (2) [below right of=1] {$5$};
    \node[main] (3) [above right of=2] {$1$};
    \node[main] (4) [below left of=2] {$3$};
    \node[main] (5) [below right of=2] {$4$};

    \draw (1) -- (2);
    \draw (2) -- (3);
    \draw (2) -- (4);
    \draw (2) -- (5);
    \draw (1) -- (3);
    \draw (4) -- (5);
  \end{tikzpicture}

  \subsection{Automorfizm}
  Automorfizm to taki izomorfizm, którego wynikiem jest graf początkowy. 

  \subsection{Formy kanoniczne grafów}
  Forma kanoniczna grafu to takie pokrycie grafu G, 
  które jest identyczne dla wszystkich grafów izomorficznych z G. Takie pokryie można zaimplementować
   jako numerowanie wierzchołków począwszy od takich o najmniejszej liczbie sąsiadów. 
   Jeżeli kilka wierzchołków ma taką samą liczbę sąsiadów to albo niższy numer otrzymuje wierzchołek, 
   który ma sąsiada (sąsiadów) z najniższym numerem albo losowy. 

  \subsection{Generowanie grafów nieizomorficznych} 
   Generowanie grafów rozpoczynamy od pojedynczego wierzchołka. Następnie powtarzamy następujące kroki:
   \begin{enumerate}
    \item Generowanie grafów nieizomorficznych \hfill 
  
    \begin{itemize}
    \item  Dodanie nowego wierzchołka  \hfill \\
    
    \item Utworzenie wszystkich możliwych zestawów krawędzi zawierających ten wierzchołek. Przy dodawaniu n-tego wierzchołka mamy 
      
    \item Sprawdzenie izomorfizmów - jeżeli wygenerowany graf jest izomorficzny z innym wygenerowanym grafem, to pozostawiamy tylko jeden z nich 
    
    \item  O ile nie osiągnęliśmy zadanej liczby wierzchołków powtórzenie kroków dla każdego wygenerowanego  
    \end{itemize}
  \end{enumerate}
  
  \begin{tikzpicture}[node distance={15mm}, main/.style = {draw, circle}] 
    \node[main] (1) {$1$};
    \node[main] (2) [below of=1] {$1$};
    \node[main] (3) [right of=2] {$2$};

    \node[main] (4) [right of=3] {$1$};
    \node[main] (5) [right of=4] {$2$};

    \draw (2) -- (3);
  \end{tikzpicture}

  \begin{tikzpicture}[node distance={15mm}, main/.style = {draw, circle}] 
    \node[main] (1) {$1$};
    \node[main] (2) [right of=1] {$2$};
    \node[main] (3) [below of=1] {$3$};

    \draw (1) -- (2);
  \end{tikzpicture} \vspace{5mm} 
  \begin{tikzpicture}[node distance={15mm}, main/.style = {draw, circle}] 
    \node[main] (1) {$1$};
    \node[main] (2) [right of=1] {$2$};
    \node[main] (3) [below of=1] {$3$};

    \draw (1) -- (2);
    \draw (1) -- (3);
  \end{tikzpicture} \vspace{5mm} 
  \begin{tikzpicture}[node distance={15mm}, main/.style = {draw, circle}] 
    \node[main] (1) {$1$};
    \node[main] (2) [right of=1] {$2$};
    \node[main] (3) [below of=1] {$3$};

    \draw (1) -- (2);
    \draw (1) -- (3);
    \draw (2) -- (3);
  \end{tikzpicture} \vspace{5mm} 
  \begin{tikzpicture}[node distance={15mm}, main/.style = {draw, circle}] 
    \node[main] (1) {$1$};
    \node[main] (2) [right of=1] {$2$};
    \node[main] (3) [below of=1] {$3$};

  \end{tikzpicture} \vspace{5mm}
  \begin{tikzpicture}[node distance={15mm}, main/.style = {draw, circle}] 
    \node[main] (1) {$1$};
    \node[main] (2) [right of=1] {$2$};
    \node[main] (3) [below of=1] {$3$};

    \draw (3) -- (2);
  \end{tikzpicture} \vspace{5mm} 
  \begin{tikzpicture}[node distance={15mm}, main/.style = {draw, circle}] 
    \node[main] (1) {$1$};
    \node[main] (2) [right of=1] {$2$};
    \node[main] (3) [below of=1] {$3$};

    \draw (3) -- (2);
    \draw (1) -- (3);
  \end{tikzpicture}

  Jak łatwo zauważyć graf 1 i graf 5 oraz graf 2 i graf 6 są parami izomorficzne, więc możne odrzucić po jednym z każdej pary, zmniejszając liczbę grafów rozważanych w kolejnych krokach.  

   \subsection{Generowanie grafów Ramseya}
   Graf jest ramseyowski jeżeli nie zawiera k-kliki lub l-zbioru niezależnego dla danych k i l. 

   Generowanie takich grafów wygląda tak samo jak generowanie grafów dowolnych, ale zawiera dodatkowy krok miedzy 3 a 4 - sprawdzenie czy graf nie zawiera kliki lub zbioru niezależnego stopnia który zaburzył by jego ramseyowskość.  
   
\section{Twierdzenie Ramseya}

Twierdzenie Ramseya mówi o konieczności pojawienia się pewnych układów w pozornym chaosie co oznacza że każda większa struktura będzie zawierała jakąś podstrukturę. Zagadnienie można łatwo przedstawić posługując się teorią grafów, dla uproszczenia zostanie użyte kolorowanie dwoma kolorami.

\begin{theorem}
Niech r $\in$ N. Istnieje takie n $\in$ N gdzie dla każdego 2-kolorowego $\mathit{K}_{n}$ grafu znajdzie się jednokolorowy podgraf $\mathit{K}_{r}$ w $\mathit{K}_{n}$.
\end{theorem}

Z powyższego twierdzenia wynika, że będziemy omawiać dwukolorowe struktury gdzie przytoczona sytuacja zachodzi. 

\begin{definition}[Liczba Ramseya]
Liczba Ramseya, wyrażana jako R(r,b), gdzie r oraz b są dowolnymi liczbami całkowitymi, jest najmniejszą możliwą liczbą n dla grafu $\mathit{K}_{n}$. Używając dwóch kolorów np. czerwonego i niebieskiego, dowolnie kolorując podany graf uzyskamy czerwony podgraf $\mathit{K}_{r}$ lub niebieski podgraf $\mathit{K}_{b}$.
\end{definition}

\section{Historia liczby i Twierdzenia Ramseya}

W 1930 roku zostało opublikowane działo Franka Plumptona Ramseya ''On a Problem of Formal Logic'', które posłużyło jako podstawę do teorii którą dzisiaj znamy jak Teoria Ramseya. 


\subsection{Wartości liczb Ramseya}

\begin{enumerate}
  \item Trywialne wartości liczby Ramseya \hfill 

  \begin{itemize}
  \item R(1,k) = R(k,1) = 1 \hfill \\
  W przypadku gdy jeden z parametrów wynosi 1 aby spełnić warunek wystarczy jeden wierzchołek.  Jednokolorowy graf $\mathit{K}_{1}$ jest pojedynczym wierzchołkiem i spełnia zarówno warunek dla R(1,b) oraz R(r,1).
  
   \item R(2,k) = R(k,2) = k \hfill \\
	W przypadku gdy jeden z parametrów wynosi 2 nie możemy postąpić analogicznie jak w poprzednim przykładzie, graf $\mathit{K}_{2}$ nie spełni warunku gdy k > 2 dla kolorowania jednym kolorem. Tak samo każdy graf pełny o rozmiarze mniejszym niż k zostanie odrzucony w sytuacji gdy zostanie użyty k kolor aby pokolorować go w jednolity sposób. Dlatego też liczba wierzchołków w grafie musi wynosić k co zawsze spełni jeden z dwóch warunków, przypadek gdy wszystkie krawędzie zostaną pokolorowane jednym kolorem lub gdy chociaż jedna krawędź będzie drugiego koloru.
  	
  \end{itemize}
 
  \item R(3,3)=6 \hfill \\ \\
 
  \begin{figure}[h]
  \centering
  \end{figure}
    Aby udowodnić że R(3,3) = 6 przeanalizujmy kolorowanie grafy pełnego o 6 wierzchołkach. Po wybraniu dowolnego wierzchołka i kolorując wychodzące z niego krawędzie co najmniej 3 z nich będą miały wspólny kolor, skupiając się na jednym kolorze przykładowo czerwonym, krawędzie to połączone są z trzema innymi wierzchołkami.
  \begin{figure}[h]
  \centering
  \end{figure}

\end{enumerate}


\end{document}
