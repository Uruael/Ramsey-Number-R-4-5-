\documentclass[11pt]{article}
 
\usepackage[T1]{fontenc}
\usepackage[polish]{babel}
\usepackage[utf8]{inputenc}
\usepackage{lmodern}
\usepackage{array}
\usepackage{mathtools}
\usepackage{tikz} 
\usetikzlibrary{graphs,graphs.standard,quotes}
\selectlanguage{polish}
\usepackage{graphicx}
\newtheorem{theorem}{Twierdzenie} 
\newtheorem{definition}{Definicja}[section]
\graphicspath{ {./im/} }




\begin{document}

\section{Podstawy teoretyczne}

  Przed omówieniem tematu naszej pracy, należy przedstawić kilka pojęć z teorii grafów, bez których zrozumienia, nie jest możliwe wyznaczanie liczb Ramseya.

  \subsection{Izomorfizm}
  Izomorfizm jest to takie przekształcenie grafu, które zachowuje wszystkie jego właściwości. Innymi słowy dwa grafy są izomorficzne, jeżeli ich wierzchołki można nazwać w taki sposób, aby sąsiadami odpowiadających sobie wierzchołków w obu grafach miały takie same zbiory sąsiadów. 

  \begin{tikzpicture}[node distance={15mm}, main/.style = {draw, circle}] 
    \node[main] (1) {$1$};
    \node[main] (2) [below right of=1] {$5$};
    \node[main] (3) [above right of=2] {$2$};
    \node[main] (4) [below left of=2] {$4$};
    \node[main] (5) [below right of=2] {$3$};

    \draw (1) -- (2);
    \draw (2) -- (3);
    \draw (2) -- (4);
    \draw (2) -- (5);
    \draw (1) -- (3);
    \draw (4) -- (5);
  \end{tikzpicture}

  Przekształćmy ten graf zamieniając wierzchołek 1 z 2 oraz 3 z 4:

  \begin{tikzpicture}[node distance={15mm}, main/.style = {draw, circle}] 
    \node[main] (1) {$2$};
    \node[main] (2) [below right of=1] {$5$};
    \node[main] (3) [above right of=2] {$1$};
    \node[main] (4) [below left of=2] {$3$};
    \node[main] (5) [below right of=2] {$4$};

    \draw (1) -- (2);
    \draw (2) -- (3);
    \draw (2) -- (4);
    \draw (2) -- (5);
    \draw (1) -- (3);
    \draw (4) -- (5);
  \end{tikzpicture}

  \subsection{Automorfizm}
  Automorfizm to taki izomorfizm, którego wynikiem jest graf początkowy. 

  \subsection{Formy kanoniczne grafów}
  Forma kanoniczna grafu to takie pokrycie grafu G, 
  które jest identyczne dla wszystkich grafów izomorficznych z G. Takie pokryie można zaimplementować
   jako numerowanie wierzchołków począwszy od takich o najmniejszej liczbie sąsiadów. 
   Jeżeli kilka wierzchołków ma taką samą liczbę sąsiadów to albo niższy numer otrzymuje wierzchołek, 
   który ma sąsiada (sąsiadów) z najniższym numerem albo losowy. 

  \subsection{Generowanie grafów nieizomorficznych} 
   Generowanie grafów rozpoczynamy od pojedynczego wierzchołka. Następnie powtarzamy następujące kroki:
   \begin{enumerate}
    \item Generowanie grafów nieizomorficznych \hfill 

    \begin{itemize}
    \item  Dodanie nowego wierzchołka  \hfill \\

    \item Utworzenie wszystkich możliwych zestawów krawędzi zawierających ten wierzchołek. Przy dodawaniu n-tego wierzchołka mamy 

    \item Sprawdzenie izomorfizmów - jeżeli wygenerowany graf jest izomorficzny z innym wygenerowanym grafem, to pozostawiamy tylko jeden z nich 

    \item  O ile nie osiągnęliśmy zadanej liczby wierzchołków powtórzenie kroków dla każdego wygenerowanego  
    \end{itemize}
  \end{enumerate}

  \begin{tikzpicture}[node distance={15mm}, main/.style = {draw, circle}] 
    \node[main] (1) {$1$};
    \node[main] (2) [below of=1] {$1$};
    \node[main] (3) [right of=2] {$2$};

    \node[main] (4) [right of=3] {$1$};
    \node[main] (5) [right of=4] {$2$};

    \draw (2) -- (3);
  \end{tikzpicture}

  \begin{tikzpicture}[node distance={15mm}, main/.style = {draw, circle}] 
    \node[main] (1) {$1$};
    \node[main] (2) [right of=1] {$2$};
    \node[main] (3) [below of=1] {$3$};

    \draw (1) -- (2);
  \end{tikzpicture} \vspace{5mm} 
  \begin{tikzpicture}[node distance={15mm}, main/.style = {draw, circle}] 
    \node[main] (1) {$1$};
    \node[main] (2) [right of=1] {$2$};
    \node[main] (3) [below of=1] {$3$};

    \draw (1) -- (2);
    \draw (1) -- (3);
  \end{tikzpicture} \vspace{5mm} 
  \begin{tikzpicture}[node distance={15mm}, main/.style = {draw, circle}] 
    \node[main] (1) {$1$};
    \node[main] (2) [right of=1] {$2$};
    \node[main] (3) [below of=1] {$3$};

    \draw (1) -- (2);
    \draw (1) -- (3);
    \draw (2) -- (3);
  \end{tikzpicture} \vspace{5mm} 
  \begin{tikzpicture}[node distance={15mm}, main/.style = {draw, circle}] 
    \node[main] (1) {$1$};
    \node[main] (2) [right of=1] {$2$};
    \node[main] (3) [below of=1] {$3$};

  \end{tikzpicture} \vspace{5mm}
  \begin{tikzpicture}[node distance={15mm}, main/.style = {draw, circle}] 
    \node[main] (1) {$1$};
    \node[main] (2) [right of=1] {$2$};
    \node[main] (3) [below of=1] {$3$};

    \draw (3) -- (2);
  \end{tikzpicture} \vspace{5mm} 
  \begin{tikzpicture}[node distance={15mm}, main/.style = {draw, circle}] 
    \node[main] (1) {$1$};
    \node[main] (2) [right of=1] {$2$};
    \node[main] (3) [below of=1] {$3$};

    \draw (3) -- (2);
    \draw (1) -- (3);
  \end{tikzpicture}

  Jak łatwo zauważyć graf 1 i graf 5 oraz graf 2 i graf 6 są parami izomorficzne, więc możne odrzucić po jednym z każdej pary, zmniejszając liczbę grafów rozważanych w kolejnych krokach.  

   \subsection{Generowanie grafów Ramseya}
   Graf jest ramseyowski jeżeli nie zawiera k-kliki lub l-zbioru niezależnego dla danych k i l. 

   Generowanie takich grafów wygląda tak samo jak generowanie grafów dowolnych, ale zawiera dodatkowy krok miedzy 3 a 4 - sprawdzenie czy graf nie zawiera kliki lub zbioru niezależnego stopnia który zaburzył by jego ramseyowskość.  

\section{Twierdzenie Ramseya}

Twierdzenie Ramseya mówi o konieczności pojawienia się pewnych układów w pozornym chaosie co oznacza że każda większa struktura będzie zawierała jakąś podstrukturę. Zagadnienie można łatwo przedstawić posługując się teorią grafów, dla uproszczenia zostanie użyte kolorowanie dwoma kolorami.

\begin{theorem}
Niech r $\in$ N. Istnieje takie n $\in$ N gdzie dla każdego 2-kolorowego $\mathit{K}_{n}$ grafu znajdzie się jednokolorowy podgraf $\mathit{K}_{r}$ w $\mathit{K}_{n}$.
\end{theorem}

Z powyższego twierdzenia wynika, że będziemy omawiać dwukolorowe struktury gdzie przytoczona sytuacja zachodzi. 

\begin{definition}[Liczba Ramseya]
Liczba Ramseya, wyrażana jako R(r,b), gdzie r oraz b są dowolnymi liczbami całkowitymi, jest najmniejszą możliwą liczbą n dla grafu $\mathit{K}_{n}$. Używając dwóch kolorów np. czerwonego i niebieskiego, dowolnie kolorując podany graf uzyskamy czerwony podgraf $\mathit{K}_{r}$ lub niebieski podgraf $\mathit{K}_{b}$.
\end{definition}

\section{Historia liczby i Twierdzenia Ramseya}

W 1928 roku zostało opublikowane działo Franka Plumptona Ramseya ''On a Problem of Formal Logic'', które posłużyło jako podstawę do teorii którą dzisiaj znamy jak Teoria Ramseya. 


\subsection{Twierdzenie Van der Waerden's}
Twierdzenie opublikowane przez Van der Waerdena w 1927 roku, przed powstaniem Twierdzenia Ramseya lecz uważana za jedną z jaj gałęzi. 

\begin{theorem}
Dla dowolnej liczby całkowitej r oraz k istnieje taka liczba N która określa zbiór \{1, 2, 3, ..., N\} który jest pokolorowany na r różnych kolorów, z przynajmniej k liczbami całkowitymi w ciągu arytmetycznym które są tego samego koloru.
\end{theorem}

\subsection{Paul Erd\"os i Happy ending problem}

Problem zaprezentowany przez Paula Erd\"osa w 1933 roku.

\begin{theorem}
Dowolny zbiór 5 punktów w przestrzeni zawiera podzbiór 4 punktów które formują wielokąt wypukły.
\end{theorem}

Twierdzenie to zostało uogólnione w 1935 roku przez George Szekeres oraz Paula Erd\"osa

\begin{theorem}
Dla dowolnej liczby całkowitej N, każdy dowolnie duży i skończony zbiór punktów zawiera podzbiór składający się z N punktów który tworzy wielokąt wypukły.
\end{theorem}

Prace nad Happy Ending problem sprawiły że Paul Erd\"os natrafił na publikacje Ramseya z 1928 roku. Spowodowało to że Erd\"os rozpoczoł prace nad liczbami Raseya, co przyczyniło się do rozwoju tej teorii.

\subsection{Wartości liczb Ramseya}

\begin{enumerate}
  \item Trywialne wartości liczby Ramseya \hfill 
  
  \begin{itemize}
  \item R(1,k) = R(k,1) = 1 \hfill \\
  W przypadku gdy jeden z parametrów wynosi 1 aby spełnić warunek wystarczy jeden wierzchołek.  Jednokolorowy graf $\mathit{K}_{1}$ jest pojedynczym wierzchołkiem i spełnia zarówno warunek dla R(1,b) oraz R(r,1).
  
   \item R(2,k) = R(k,2) = k \hfill \\
	W przypadku gdy jeden z parametrów wynosi 2 nie możemy postąpić analogicznie jak w poprzednim przykładzie, graf $\mathit{K}_{2}$ nie spełni warunku gdy k > 2 dla kolorowania jednym kolorem. Tak samo każdy graf pełny o rozmiarze mniejszym niż k zostanie odrzucony w sytuacji gdy zostanie użyty k kolor aby pokolorować go w jednolity sposób. Dlatego też liczba wierzchołków w grafie musi wynosić k co zawsze spełni jeden z dwóch warunków, przypadek gdy wszystkie krawędzie zostaną pokolorowane jednym kolorem lub gdy chociaż jedna krawędź będzie drugiego koloru.
  	
  \end{itemize}

  
 
  \item R(3,3)=6 \hfill \\ \\
  R(3,3) jest pierwszym nietrywialnym przykładem liczby Ramseya, lecz nadal na tyle małą aby łatwo móc ją wyznaczyć. Łatwo można wykluczyć $\mathit{K}_{3}$, $\mathit{K}_{4}$ oraz $\mathit{K}_{5}$ za pomocą następującego pokolorowania krawędzi.
  
  \begin{figure}[h]
  \centering
  \includegraphics[width=0.35\textwidth]{k5}
  \end{figure}
  

  
Aby udowodnić że R(3,3) = 6 przeanalizujmy kolorowanie grafu pełnego o 6 wierzchołkach. 
  \\
  \begin{figure}[h]
  \centering
  \includegraphics[width=0.75\textwidth]{k6}
  \end{figure}  
  \\
Po wybraniu dowolnego wierzchołka i kolorując wychodzące z niego krawędzie co najmniej 3 z nich będą miały wspólny kolor, skupiając się na jednym kolorze przykładowo czerwonym, krawędzie te połączone są z trzema innymi wierzchołkami. Patrząc na trzy wierzchołki do których zostały poprowadzone krawędzie jednego koloru, łatwo zauważyć, że aby uniknąć powstania trójkąta czerwonego należy połączyć te wierzchołki kolorem niebieskim lecz robiąc to powstanie klika o rozmiarze trzy koloru niebieskiego. Dowodzi to że R(3,3) = 6.
    



\item Inne liczby Ramseya \hfill \\

 Udowodnienie wartości pozostałych liczb Ramseya zostanie pominięte, gdyż stopień skomplikowania dowodu rośnie wraz z ilością wierzchołków, nie istnieje żaden znany sposób na określenie dokładnej wartości tej liczby, oraz wyznaczenie dokładnej wartości często jest na tyle trudne że istnieje jedynie jej bliższe oszacowanie. Poniższa tabela prezentuje dokładne wartości lub górne i dolne granice dla dwukolorowych liczb Ramseya R(k,l) k<10, l<10 (wartości dla k i l równego 2 albo 1 zostały opisane wcześniej)
 
\hfill 

\resizebox{\textwidth}{!}{%

\begin{tabular}{|l|l|l|l|l|l|l|l|l|}
\hline
k\textbackslash{}l & 3 & 4  & 5                                               & 6                                                 & 7                                                 & 8                                                  & 9                                                  & 10                                                  \\ \hline
3                  & 6 & 9  & 14                                              & 18                                                & 23                                                & 28                                                 & 36                                                 & \begin{tabular}[c]{@{}l@{}}40\\ 43\end{tabular}     \\ \hline
4                  &   & 18 & 25                                              & \begin{tabular}[c]{@{}l@{}}35\\ 41\end{tabular}   & \begin{tabular}[c]{@{}l@{}}49\\ 61\end{tabular}   & \begin{tabular}[c]{@{}l@{}}56\\ 84\end{tabular}    & \begin{tabular}[c]{@{}l@{}}73\\ 115\end{tabular}   & \begin{tabular}[c]{@{}l@{}}92\\ 149\end{tabular}    \\ \hline
5                  &   &    & \begin{tabular}[c]{@{}l@{}}43\\ 49\end{tabular} & \begin{tabular}[c]{@{}l@{}}58\\ 87\end{tabular}   & \begin{tabular}[c]{@{}l@{}}80\\ 143\end{tabular}  & \begin{tabular}[c]{@{}l@{}}101\\ 216\end{tabular}  & \begin{tabular}[c]{@{}l@{}}125\\ 316\end{tabular}  & \begin{tabular}[c]{@{}l@{}}143\\ 442\end{tabular}   \\ \hline
6                  &   &    &                                                 & \begin{tabular}[c]{@{}l@{}}102\\ 165\end{tabular} & \begin{tabular}[c]{@{}l@{}}113\\ 298\end{tabular} & \begin{tabular}[c]{@{}l@{}}127\\ 495\end{tabular}  & \begin{tabular}[c]{@{}l@{}}169\\ 780\end{tabular}  & \begin{tabular}[c]{@{}l@{}}179\\ 1171\end{tabular}  \\ \hline
7                  &   &    &                                                 &                                                   & \begin{tabular}[c]{@{}l@{}}205\\ 540\end{tabular} & \begin{tabular}[c]{@{}l@{}}216\\ 1031\end{tabular} & \begin{tabular}[c]{@{}l@{}}233\\ 1713\end{tabular} & \begin{tabular}[c]{@{}l@{}}289\\ 2826\end{tabular}  \\ \hline
8                  &   &    &                                                 &                                                   &                                                   & \begin{tabular}[c]{@{}l@{}}282\\ 1870\end{tabular} & \begin{tabular}[c]{@{}l@{}}317\\ 3583\end{tabular} & \begin{tabular}[c]{@{}l@{}}?\\ 6090\end{tabular}    \\ \hline
9                  &   &    &                                                 &                                                   &                                                   &                                                    & \begin{tabular}[c]{@{}l@{}}565\\ 6588\end{tabular} & \begin{tabular}[c]{@{}l@{}}580\\ 12677\end{tabular} \\ \hline
10                 &   &    &                                                 &                                                   &                                                   &                                                    &                                                    & \begin{tabular}[c]{@{}l@{}}798\\ 23556\end{tabular} \\ \hline
\end{tabular}%
}

\hfill 

Powodem dla podawania granicy dolnej oraz górnej jest, jak wspomniano wcześniej, brak uniwersalnej i opłacalnej formuły do określenia dokładnej wartości. Przykładowo dla R(4,6) gdybyśmy chcieli sprawdzić dolną granicą 35, trzeba sprawdzić wszystkie dwukolorowania $\mathit{K}_{35}$, który ma ${35\choose 2}$ = 595 krawędzi. Istnieje więc $2^{595} \approx 1.296 * 10^{179}$ różnych możliwości na pokolorowanie tego grafu. Dlatego w obecnych czasach nie jest możliwe potraktowanie tego problemu używając podejścia naiwnego. \par

\item Granice liczby Ramseya \hfill \\

Granica górna może być łatwo wyliczona stosując nierówność $R(r,b) \le R(r-1,b)+R(r,b-1)$. Nie jest to jednak zadowalający wynik ani sposób na wyznaczanie górnego limitu. Poprzednie wartości liczb Ramseya mogą nie być znane oraz sama granica przy znanych wcześniejszych wartościach nie jest najbardziej optymalną. Wzór jawny który opisuje wcześniej podany przypadek to: $R(r,b) \le {r+b-2\choose r-1}$. Przytoczona górna granica jest granicą naiwną. Granica dolna jest wyznaczana z użyciem metod probabilistycznych. Paul Erdős jako pierwszy w 1947 roku zaprezentował dowód z użyciem metod probabilistycznych na granicę dolną dla liczb R(k,k). Metoda ta opierała się na wykazaniu że w losowo pokolorowanym grafie $\mathit{K}_{n}$ prawdopodobieństwo znalezienia jednokolorowego grafu $\mathit{K}_{k}$ jest mniejsza od 1 dla pewnej wartości.

\end{enumerate}

\section{Artykuł R(4,5)=25}

\section{Sklejanie grafów}
Efektywne generowanie dużych grafów jest bardzo wymagające zarówno czasowo jak i pamięciowo. Dlatego stosujemy technikę nazywaną sklejaniem grafów. 

\subsection{Idea sklejania}
Robimy to poprzez wygenerowanie wszyskich nieizomorficznych grafów G, rzędów 7-13, spełniających R(3,5) oraz wszyskich
nieizomorficznych grafów H, rzędów 11-17, spełniających R(4, 4).'

Następnie dokonujemy prób sklejania grafów z dwóch grup - kryterium przy parowaniu jest suma rzędów, która my wynosić 24. 

\subsection{Algorytm sklejania}
Zdefiniujmy prawdopodobny stożek (ang. feasible cone) jako podzbiór wszyskich wierzchołków H, który nie tworzy kliki 3.
Dla przeciętnego grafu (4, 4, 14) znajdziemy około 4000 takich prawdopodobnych stożków. 
Tak zdefiniowane stożki należy przyporządkować do wierzchołków  grafu G, tak aby w wyniku ich połączenia nie powtała kilka 4 lub zbiór niezależny 5.
Możnaby podjąć próbę zrobienia tego w sposób naiwny, ale byłoby zdecydowanie zbyt wolne, dla tak dużej ilości rozpatrywanych grafów.

Zdefiniujmy 3 funkcje działające na X, będącym podzbiorem wierzchołków grafu H, generujące podzbiór grafu H. 
Funcja H1 będzie wybierała wszystkich sąsiadów wierzchołków ze zbioru X.
H2 wybiera wszystkie wierzchołki, które nie sąsiadują z jednym lub większą ilością wierzchołków X. 
H3 wybiera wszystkie wierzchołki, które nie sąsiadują z dwoma lub większą ilością wierzchołków X.
\begin{itemize}
    
  \item   H1(X) = \{ w $\in$ VH | vw $\in$ EX dla jakiegoś v $\in$ X \} 
  
  \item   H2(X) = \{ w $\in$ VH | vw $\notin$ EX dla jakiegoś v $\notin$ X\}
  
  \item   H3(X) = \{ w $\in$ VH | { u, v, w } jest zbiorem niezależnym dla jakichś v, u $\in$ X\} 
  
\end{itemize}

\subsection{Wynik sklejania}


\section{Implementacja i eksperymenty}
Nasz program został zaimplementowany w języku C z wykorzystaniem biblioteki Nauty, która umożliwia łatwe obliczeniowo wykrywanie izomorfizów.
Autor biblioteki jest profesor McKay. 

\begin{thebibliography}{9}
  \bibitem{mainpaper} 
  Brendan D. McKay, Stanisław P. Radziszowski. 
  \textit{R(4,5) = 25}. 
  
  \bibitem{nauty} 
  Brendan D. McKay.
  \textit{nauty user's guide (version 2.4)}.  
  Department of Computer Science
  Australian National University
  Canberra ACT 0200, Australia
  November 4, 2009

  \bibitem{smallramsey} 
  Stanisław P. Radziszowski.
  \textit{Small Ramsey Numbers}.  
  Department of Computer Science Rochester Institute of Technology Rochester, NY 14623
  June 11, 1994

  \bibitem{cykl4} 
  Janusz Dybizbański.
  \textit{Liczby Ramseya z cyklem C4}.  
  Uniwersytet Warszawski
  Wydział Matematyki, Informatyki i Mechaniki
  Listopad 2013

  \bibitem{numbers} 
  Christos Nestor Chachamis.
  \textit{Ramsey Numbers}.  
  May 13, 2018

  \bibitem{theory} 
  Lane Barton IV
  \textit{Ramsey Theory}.  
  May 13, 2016

  \bibitem{generation} 
  Brendan D. McKay
  \textit{ISOMORPH-FREE EXHAUSTIVE GENERATION}.  
  May 13, 2016

  \end{thebibliography}

\end{document}
