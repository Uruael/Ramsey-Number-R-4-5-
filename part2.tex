\documentclass[11pt]{article}
 
\usepackage[T1]{fontenc}
\usepackage[polish]{babel}
\usepackage[utf8]{inputenc}
\usepackage{algorithm}
\usepackage{algorithmic}
\usepackage{subcaption}
\usepackage{amsfonts} 

\usepackage{float}
\usepackage{subcaption}
\usepackage{lmodern}
\usepackage{array}
\usepackage{mathtools}
\usepackage{tikz} 
\usetikzlibrary{graphs,graphs.standard,quotes}
\selectlanguage{polish}
\usepackage{graphicx}
\newtheorem{theorem}{Twierdzenie} 
\newtheorem{definition}{Definicja}[section]
\graphicspath{ {./im/} }

\title{R(4,5) = 25 - powtórzenie eksperymentu}
\author{Krzysztof Karpusiewicz, Mateusz Kołakowski, Jan Kwarciński}

\begin{document}

\tableofcontents
\pagebreak

\section{Podstawy teoretyczne}

  Przed omówieniem tematu naszej pracy, należy przedstawić kilka pojęć z teorii grafów, bez których zrozumienia, nie jest możliwe wyznaczanie liczb Ramseya.

  \subsection{Graf}
  \begin{definition}
    Graf nieskierowany to para G=(V, E), gdzie V to zbiór elementów nazywanych wierzchołkami a E to zbiór par wierzchołków, którego elementy są nazywane krawędziami.
  \end{definition}

  Dla potrzeb naszej pracy, będziemy zakładać, że gdy mówimy o grafie, mamy na myśli graf nieskierowany i prosty (brak duplikatów par w zbiorze E).
  
  \subsection{Rząd grafu}
  \begin{definition}
    Liczbę elementów zbioru V w grafie G=(V, E) nazywamy rzędem grafu. 
  \end{definition}

  Rząd grafu to liczba jego wierzchołków. 

  \subsection{Sąsiedztwo wierzchołków}
  \begin{definition}
    W grafie G (V, E) wierzchołki u, v $\in$ V sąsiadują wtedy i tylko wtedy gdy (u, v) $\in$ E.    
  \end{definition}
  W grafach nieskierowanych sąsiedztwo jest relacją symetryczną. Jest nieprzechodnia - z sąsiedztwa u i v oraz v i w nie wynika sąsiedztwo u i w. \linebreak

  \begin{figure}[h]
    \centering
    \begin{tikzpicture}[node distance={15mm}, main/.style = {draw, circle}] 
      \node[main] (1) {$1$};
      \node[main] (2) [below right of=1] {$5$};
      \node[main] (3) [above right of=2] {$2$};
      \node[main] (4) [below left of=2] {$4$};
      \node[main] (5) [below right of=2] {$3$};

      \draw (1) -- (2);
      \draw (2) -- (3);
      \draw (2) -- (4);
      \draw (2) -- (5);
      \draw (1) -- (3);
      \draw (4) -- (5);
    \end{tikzpicture}
    \caption{Graf, w którym wierzchołek 1 sąsiaduje z 2 i 5; 2 z 1 i 5; 3 z 4 i 5; 4 z 3 i 5; 5 z 1, 2, 3 i 4 }
  \end{figure}

  \subsection{Izomorfizm}

  \begin{definition}
    Izomorfizm grafów G i H to taka biekcja f:V(G) -> V(H), że 2 wierchołki u i v sąsiadują w G tylko jeżeli f(u) i f(v) sąsiadują w H.
  \end{definition}

  Innymi słowy, izomorfizm jest to takie przekształcenie grafu, które zachowuje wszystkie jego właściwości. Grafy, dla których można znaleźć izomorfizm są izomorficzne. Innymi słowy, dwa grafy są izomorficzne, jeżeli ich wierzchołki można nazwać w taki sposób, 
  aby sąsiadami odpowiadających sobie wierzchołków w obu grafach miały dokładnie takie same zbiory sąsiadów.
  

  Przekształćmy graf z rysunku 1 zamieniając wierzchołek 1 z 2 oraz 3 z 4:
  \begin{figure}[h]
    \centering
    \begin{tikzpicture}[node distance={15mm}, main/.style = {draw, circle}] 
      \node[main] (1) {$2$};
      \node[main] (2) [below right of=1] {$5$};
      \node[main] (3) [above right of=2] {$1$};
      \node[main] (4) [below left of=2] {$3$};
      \node[main] (5) [below right of=2] {$4$};

      \draw (1) -- (2);
      \draw (2) -- (3);
      \draw (2) -- (4);
      \draw (2) -- (5);
      \draw (1) -- (3);
      \draw (4) -- (5);
    \end{tikzpicture}
    \caption{Mimo przekształcenia graf zachował swoje własności}
  \end{figure}

  Jak widać, wierzchołki o tych samych numerach zachowały swoje zbiory sąsiadów. 

  Warto zauważyć, że na nasz użytek grafy izomorficzne nie są niezależnie użyteczne. Wszystkie własności opisane w tej pracy są przez izomorfizmy niezmienione, a więc na nasz użytek dwa grafy izomorficzne to efektywnie ten sam graf.


  \subsection{Automorfizm}

  \begin{definition}
    Automorfizm grafu G to taka permutacja f zbioru wierzchołków G, że para wierzchołków u, v jest sąsiedująca tylko jeżeli f(u) i f(v) sąsiadują. 
  \end{definition}
  Automorfizm to izomorfizm z grafu początkowego do tego samego grafu. 

  \subsection{Formy kanoniczne grafów}

  \begin{definition}
    Forma kanoniczna C(G) grafu G to taki graf izomorficzny do G, że każdy graf izomorficzny do G ma taką samą formę kanoniczną. 
  \end{definition}
  Forma kanoniczna grafu to takie pokrycie grafu G, 
  które jest identyczne dla wszystkich grafów izomorficznych z G. Takie pokrycie można zaimplementować
   jako numerowanie wierzchołków począwszy od takich o najmniejszej liczbie sąsiadów. 
   Jeżeli kilka wierzchołków ma taką samą liczbę sąsiadów to albo niższy numer otrzymuje wierzchołek, 
   który ma sąsiada (sąsiadów) z najniższym numerem albo losowy. 

   \begin{algorithm}
    \caption{Sprowadzanie grafu do formy kanonicznej}
    \begin{algorithmic}
    \STATE formaKanoniczna[liczbaWierzchołków(graf)]
    \FORALL{formaKanoniczna}
      \WHILE{$min \neq 0$}
        \STATE Jiji
      \ENDWHILE
    \ENDFOR
    \WHILE{$ilośćWierzchołków(grafy) < n$}
      \STATE dodajWierzchołek(grafy)
      \STATE usunIzomorfizmy(grafy)
    \ENDWHILE
    \end{algorithmic}
    \end{algorithm}

    \subsection{Orbity}
    \begin{definition}
      Orbita w grafie G to taki podzbiór jego wierzchołków, których wszystkie permutacje są izomorfizmami. 
    \end{definition}

    Wszystkie wierzchołki wewnątrz orbity mogą zostać dowolnie zamienione, nie zmieniając właściwości grafu. Ich zamiana jest automorfizmem.
  Przykład: 
    \begin{figure}[H]
      \centering
      \begin{tikzpicture}[node distance={15mm}, main/.style = {draw, circle}] 
        \node[main] (1) {$1$};  
        \node[main] (3) [right of=1] {$3$};
        \node[main] (2) [above of=3] {$2$};
        \node[main] (4) [below of=3] {$4$};
        \node[main] (5) [right of=3] {$5$};

        \draw (1) -- (3);
        \draw (1) -- (2);
        \draw (1) -- (4);
        \draw (5) -- (2);
        \draw (5) -- (3);
        \draw (5) -- (4);
        \end{tikzpicture}
        \caption{Orbita pierwsza: 1, 5 
        Orbita druga: 2, 3, 4}
    \end{figure}

   \subsection{Klika}
   \begin{definition}
    Klika K w grafie G jest takim podzbiorem wierzchołków G, że dla każdej pary wierzchołków u, v $\in$ K, u i v są sąsiednie. 
   \end{definition}
   
   W uproszczeniu, klika to podzbiór wierzchołków grafu, z których każdy jest połaczony z każdem innym wiechołkiem tego podzbioru.
   \begin{figure}[H]
   \centering
    \begin{tikzpicture}[node distance={15mm}, main/.style = {draw, circle}] 
      \node[main] (1) {$1$};
      \node[main] (3) [below right of=1] {$3$};
      \node[main] (2) [below left of=3] {$2$};
      \node[main] (4) [below right of=3] {$4$};
      \node[main] (5) [above right of=3] {$5$};
      
      \draw (3) -- (2);
      \draw (1) -- (3);
      \draw (1) -- (2);
      \draw (2) -- (4);
      \draw (4) -- (5);
      \draw (1) -- (5);
     \end{tikzpicture}
     \caption{Wierzchołki 1, 2, 3 tworzą klikę stopnia 3 - $K_3$ }
  \end{figure}

  \subsection{Zbiór niezależny}
  \begin{definition}
    Zbiór niezależny N w grafie G to taki podzbiór wierzchołków G, że dla każdej pary wierzchołków u, v $\in$ N, u i v nie są sąsiednie. 
  \end{definition}
  
    Zbiór niezależny można również przedstawić jako przeciwieństwo kliki.
    \begin{figure}[H]
      \centering
       \begin{tikzpicture}[node distance={15mm}, main/.style = {draw, circle}] 
         \node[main] (1) {$1$};
         \node[main] (3) [below right of=1] {$3$};
         \node[main] (2) [below left of=3] {$2$};
         \node[main] (4) [below right of=3] {$4$};
         \node[main] (5) [above right of=3] {$5$};
         
         \draw (2) -- (4);
         \draw (4) -- (5);

        \end{tikzpicture}
        \caption{Wierzchołki 1, 2, 3, 5 tworzą zbiór niezależny 4 - $N_4$ }
     \end{figure}

\section{Twierdzenie Ramseya}

Twierdzenie Ramseya mówi o konieczności pojawienia się pewnych układów w pozornym chaosie co oznacza że każda większa struktura będzie zawierała jakąś podstrukturę. Zagadnienie można łatwo przedstawić posługując się teorią grafów, dla uproszczenia zostanie użyte kolorowanie dwoma kolorami.

\begin{theorem}
Niech r $\in \mathbb{N}$. Istnieje takie n $\in \mathbb{N}$  gdzie dla każdego 2-kolorowego $\mathit{K}_{n}$ grafu znajdzie się jednokolorowy podgraf $\mathit{K}_{r}$ w $\mathit{K}_{n}$.  \cite{theory} 
\end{theorem}

Z powyższego twierdzenia wynika, że będziemy omawiać dwukolorowe struktury. 

\begin{definition}[Liczba Ramseya]
Liczba Ramseya, wyrażana jako R(r,b), gdzie r oraz b są dowolnymi liczbami całkowitymi, jest najmniejszą możliwą liczbą n dla grafu $\mathit{K}_{n}$. Używając dwóch kolorów np. czerwonego i niebieskiego, dowolnie kolorując podany graf uzyskamy czerwony podgraf $\mathit{K}_{r}$ lub niebieski podgraf $\mathit{K}_{b}$.  \cite{theory} 
\end{definition}

W kontekście tej pracy, określamy graf jako ramseyowski jeżeli nie posiada on kolorowania określonego w twierdzeniu 1. Oznacza to, że graf spełniający R(4,5) to graf, który nie posiada kliki 4 rzędu ani zbioru niezależnego 5 rzędu. Dodatkowo zapis R(4,5,n) odnosi się do grafu n wierzchołkowego spełniającego R(4,5). 
Celem tej pracy jest wykazanie, że stworzenie grafu R(4,5,25) jest niemożliwe, z czego wynika, że liczba Ramseya R(4,5) = 25.

W celu ułatwienia obliczeń i umożliwienia użycia technik generacji grafów, używamy uproszczenia reprezentacji grafowej. Zamiast wyznaczać kolorowania grafu pełnego, używamy wszystkich grafów prostych gdzie kolor krawędzi zamieniony jest na istnienie lub brak istnienia krawędzi pomiędzy parą wierzchołków.

\section{Historia liczby i Twierdzenia Ramseya}

W 1930 roku zostało opublikowane działo Franka Plumptona Ramseya ''On a Problem of Formal Logic''\cite{ramsey} , które posłużyło jako podstawę do teorii którą dzisiaj znamy jak Teoria Ramseya. 


\subsection{Twierdzenie Van der Waerden's}
Twierdzenie opublikowane przez Van der Waerdena w 1927 roku, przed powstaniem Twierdzenia Ramseya lecz uważana za jedną z jego gałęzi. 

\begin{theorem}
Dla dowolnych liczb naturalnych r oraz k istnieje taka liczba N która określa zbiór \{1, 2, 3, ..., N\} który jest pokolorowany na r różnych kolorów, z przynajmniej k liczbami naturalnymi w ciągu arytmetycznym które są tego samego koloru.\cite{theory} 
\end{theorem}

Dla przykładu, W(2,3) = 9. Zbiór o ośmiu elementach można podzielić na następujące podzbiory aby w żadnym z podzbiorów nie wystąpił szereg arytmetyczny o trzech elementach, \{1,2,3,4,5,6,7,8\} -> \{1,2,5,6\},\{3,4,7,8\}. W przypadku zbioru o dziewięciu elementach nie jest to możliwe. Dodając 9 do dowolnego podzbioru utworzony zostanie ciąg arytmetyczny o trzech elementach np. \{1,5,9\} lub \{7,8,9\}. Podobna sytuacja zajdzie dla podziału \{1,4,5,8\},\{2,3,6,7\}

\subsection{Paul Erd\"os i Teoria Ramseya}

Happy Ending problem, czyli problem zaprezentowany przez Paula Erd\"osa w 1933 roku brzmi następująco. 

\begin{theorem}
Dowolny zbiór 5 punktów w przestrzeni zawiera podzbiór 4 punktów które formują wielokąt wypukły.
\end{theorem}

Twierdzenie to zostało uogólnione w 1935 roku przez George Szekeres oraz Paula Erd\"osa

\begin{theorem}
Dla dowolnej liczby całkowitej N, każdy dowolnie duży i skończony zbiór punktów zawiera podzbiór składający się z N punktów który tworzy wielokąt wypukły. \cite{erdoshappy} 
\end{theorem}

Prace nad Happy Ending problem sprawiły że Paul Erd\"os natrafił na publikacje Ramseya z 1928 roku. Spowodowało to że Erd\"os rozpoczął prace nad liczbami Raseya, co przyczyniło się do rozwoju tej teorii.

\subsection{Wartości liczb Ramseya}

\begin{enumerate}
  \item Trywialne wartości liczby Ramseya \hfill 
  
  \begin{itemize}
  \item R(1,k) = R(k,1) = 1 \hfill \par
  W przypadku gdy jeden z parametrów wynosi 1 aby spełnić warunek wystarczy jeden wierzchołek.  Jednokolorowy graf $\mathit{K}_{1}$ jest pojedynczym wierzchołkiem i spełnia zarówno warunek dla R(1,b) oraz R(r,1). 
  
   \item R(2,k) = R(k,2) = k \hfill  \par
	W przypadku gdy jeden z parametrów wynosi 2 nie możemy postąpić analogicznie jak w poprzednim przykładzie, graf $\mathit{K}_{2}$ nie spełni warunku gdy k > 2 dla kolorowania jednym kolorem. Tak samo każdy graf pełny o rozmiarze mniejszym niż k zostanie odrzucony w sytuacji gdy zostanie użyty k kolor aby pokolorować go w jednolity sposób. Dlatego też liczba wierzchołków w grafie musi wynosić k co zawsze spełni jeden z dwóch warunków, przypadek gdy wszystkie krawędzie zostaną pokolorowane jednym kolorem lub gdy chociaż jedna krawędź będzie drugiego koloru.
  	
  \end{itemize}

  
 
  \item R(3,3)=6 \hfill \par
  R(3,3) jest pierwszym nietrywialnym przykładem liczby Ramseya, lecz nadal na tyle małą aby łatwo móc ją wyznaczyć. Łatwo można wykluczyć $\mathit{K}_{3}$, $\mathit{K}_{4}$ oraz $\mathit{K}_{5}$ za pomocą następującego pokolorowania krawędzi.


\begin{figure}[h]
  \centering
   \begin{tikzpicture}[node distance={15mm}, main/.style = {draw, circle}] 
 \node[main] (1) at (1*90:1) {$1$};
 \node[main] (2) at (2*0:1.5) {$2$};
 \node[main] (5) at (3*60:1.5) {$5$};
 \node[main] (4) at (4*60:2) {$4$};
 \node[main] (3) at (5*60:2) {$3$};

\draw[ultra thick][red] (1) -- (2);
\draw[ultra thick][red] (2) -- (3);
\draw[ultra thick][red] (3) -- (4);
\draw[ultra thick][red] (4) -- (5);
\draw[ultra thick][red] (5) -- (1);

\draw[ultra thick][blue] (1) -- (3);
\draw[ultra thick][blue] (1) -- (4);
\draw[ultra thick][blue] (2) -- (4);
\draw[ultra thick][blue] (2) -- (5);
\draw[ultra thick][blue] (3) -- (5);
    \end{tikzpicture}
    \caption{}
 \end{figure}

Powyższy rysunek pokazuje sposób kolorowania dla grafu 5-wierzchołkowego, ale wykluczenie dowolnego wierzchołka daje poprawne kolorowanie dla grafu 4-wierzchołkowego, dowolnych dwóch dla grafu 3-wierzchołkowego itd.

Aby udowodnić że R(3,3) = 6 przeanalizujmy kolorowanie grafu pełnego o 6 wierzchołkach. 


\begin{figure}[h]
  \centering
   \begin{tikzpicture}[node distance={15mm}, main/.style = {draw, circle}] 
 \node[main] (3) at (0:2) {$3$};
 \node[main] (2) at (60:2) {$2$};
 \node[main] (1) at (120:2) {$1$};
 \node[main] (6) at (180:2) {$6$};
 \node[main] (5) at (240:2) {$5$};
 \node[main] (4) at (300:2) {$4$};

\draw[ultra thick][red] (6) -- (1);
\draw[ultra thick][red] (6) -- (2);
\draw[ultra thick][red] (6) -- (3);
\draw[ultra thick][blue] (6) -- (4);
\draw[ultra thick][blue] (6) -- (5);
    \end{tikzpicture}
    \caption{}
 \end{figure}

\begin{figure}[h]
  \centering
   \begin{tikzpicture}[node distance={15mm}, main/.style = {draw, circle}] 
 \node[main] (3) at (0:2) {$3$};
 \node[main] (2) at (60:2) {$2$};
 \node[main] (1) at (120:2) {$1$};
 \node[main] (6) at (180:2) {$6$};
 \node[main] (5) at (240:2) {$5$};
 \node[main] (4) at (300:2) {$4$};

\draw[ultra thick][red] (6) -- (1);
\draw[ultra thick][red] (6) -- (2);
\draw[ultra thick][red] (6) -- (3);
\draw[ultra thick][blue] (6) -- (4);
\draw[ultra thick][blue] (6) -- (5);

\draw[ultra thick][blue] (2) -- (1);
\draw[ultra thick][blue] (3) -- (2);
\draw[ultra thick][blue] (1) -- (3);
    \end{tikzpicture}
    \caption{}
 \end{figure}



Po wybraniu dowolnego wierzchołka i pokolorowaniu wychodzących z niego krawędzie co najmniej trzy z nich będą miały wspólny kolor. Na rysunku przykładowym tym kolorem jest kolor czerwony, krawędzie tego koloru połączone są z trzema innymi wierzchołkami. Patrząc na trzy wierzchołki do których zostały poprowadzone krawędzie czerwone, łatwo zauważyć, że aby uniknąć powstania trójkąta czerwonego należy połączyć te wierzchołki kolorem niebieskim, lecz robiąc to powstanie klika o rozmiarze trzy koloru niebieskiego. Dowodzi to że R(3,3) = 6.
    



\item Inne liczby Ramseya \hfill \par

 Udowodnienie wartości pozostałych liczb Ramseya zostanie pominięte, gdyż stopień skomplikowania dowodu rośnie wraz z ilością wierzchołków, nie istnieje żaden znany łatwy obliczeniowo sposób na określenie dokładnej wartości tej liczby, oraz wyznaczenie dokładnej wartości często jest na tyle trudne że istnieje jedynie jej bliższe oszacowanie. Poniższa tabela prezentuje dokładne wartości lub górne i dolne granice dla dwukolorowych liczb Ramseya R(k,l) k<10, l<10 (wartości dla k i l równego 2 albo 1 zostały opisane wcześniej)
 
\hfill 
\begin{figure}[h]
  \centering
\resizebox{\textwidth}{!}{%
\begin{tabular}{|l|l|l|l|l|l|l|l|l|}
\hline
k\textbackslash{}l & 3 & 4  & 5                                               & 6                                                 & 7                                                 & 8                                                  & 9                                                  & 10                                                  \\ \hline
3                  & 6 & 9  & 14                                              & 18                                                & 23                                                & 28                                                 & 36                                                 & \begin{tabular}[c]{@{}l@{}}40\\ 42\end{tabular}     \\ \hline
4                  &   & 18 & 25                                              & \begin{tabular}[c]{@{}l@{}}36\\ 41\end{tabular}   & \begin{tabular}[c]{@{}l@{}}49\\ 61\end{tabular}   & \begin{tabular}[c]{@{}l@{}}59\\ 84\end{tabular}    & \begin{tabular}[c]{@{}l@{}}73\\ 115\end{tabular}   & \begin{tabular}[c]{@{}l@{}}92\\ 149\end{tabular}    \\ \hline
5                  &   &    & \begin{tabular}[c]{@{}l@{}}43\\ 49\end{tabular} & \begin{tabular}[c]{@{}l@{}}58\\ 87\end{tabular}   & \begin{tabular}[c]{@{}l@{}}80\\ 143\end{tabular}  & \begin{tabular}[c]{@{}l@{}}101\\ 216\end{tabular}  & \begin{tabular}[c]{@{}l@{}}133\\ 316\end{tabular}  & \begin{tabular}[c]{@{}l@{}}143\\ 442\end{tabular}   \\ \hline
6                  &   &    &                                                 & \begin{tabular}[c]{@{}l@{}}102\\ 165\end{tabular} & \begin{tabular}[c]{@{}l@{}}115\\ 298\end{tabular} & \begin{tabular}[c]{@{}l@{}}134\\ 495\end{tabular}  & \begin{tabular}[c]{@{}l@{}}183\\ 780\end{tabular}  & \begin{tabular}[c]{@{}l@{}}204\\ 1171\end{tabular}  \\ \hline
7                  &   &    &                                                 &                                                   & \begin{tabular}[c]{@{}l@{}}205\\ 540\end{tabular} & \begin{tabular}[c]{@{}l@{}}217\\ 1031\end{tabular} & \begin{tabular}[c]{@{}l@{}}252\\ 1713\end{tabular} & \begin{tabular}[c]{@{}l@{}}292\\ 2826\end{tabular}  \\ \hline
8                  &   &    &                                                 &                                                   &                                                   & \begin{tabular}[c]{@{}l@{}}282\\ 1870\end{tabular} & \begin{tabular}[c]{@{}l@{}}329\\ 3583\end{tabular} & \begin{tabular}[c]{@{}l@{}}343\\ 6090\end{tabular}  \\ \hline
9                  &   &    &                                                 &                                                   &                                                   &                                                    & \begin{tabular}[c]{@{}l@{}}565\\ 6588\end{tabular} & \begin{tabular}[c]{@{}l@{}}581\\ 12677\end{tabular} \\ \hline
10                 &   &    &                                                 &                                                   &                                                   &                                                    &                                                    & \begin{tabular}[c]{@{}l@{}}798\\ 23556\end{tabular} \\ \hline
\end{tabular}%
}
\end{figure}
\hfill 

Powodem dla podawania granicy dolnej oraz górnej jest, jak wspomniano wcześniej, brak uniwersalnej i opłacalnej formuły do określenia dokładnej wartości. Przykładowo dla R(4,6) gdybyśmy chcieli sprawdzić dolną granicą 36 \cite{smallramsey} , trzeba sprawdzić wszystkie dwukolorowania $\mathit{K}_{36}$, który ma ${36\choose 2}$ = 630 krawędzi. Istnieje więc $2^{630} \approx 4,4555 * 10^{189}$ różnych możliwości na pokolorowanie tego grafu. Dlatego w obecnych czasach nie jest możliwe potraktowanie tego problemu używając podejścia naiwnego. Aby podsumować problem znajdowania dokładnych warości liczb Ramseya można posłużyć się cytatem Paula Erdősa: Jeżeli kosmici najechali by ziemię i postawili ultimatum że jeżeli ludzkość nie znajdzie R(5,5) w ciągu roku, to zniszczą ziemię, najlepszym wyborem było by zebranie mocy obliczeniowej jaka aktualnie dysponuje ludzkość w celu pozyskania tej liczby. Jednak w przypadku gdy kosmici zażądali by R(6,6) najlepszym wyborem było by wypowiedzenie im wojny.\cite{aliens}  \par

\item Granice liczby Ramseya \hfill \par

Granica górna może być łatwo wyliczona stosując nierówność $R(r,b) \le R(r-1,b)+R(r,b-1)$ \cite{graniceupdown} . Nie jest to jednak zadowalający wynik ani sposób na wyznaczanie górnego limitu. Poprzednie wartości liczb Ramseya mogą nie być znane oraz sama granica przy znanych wcześniejszych wartościach nie jest najbardziej optymalną. Wzór jawny który opisuje wcześniej podany przypadek to: $R(r,b) \le {r+b-2\choose r-1}$. Przytoczona górna granica jest granicą naiwną. Granica dolna jest wyznaczana z użyciem metod probabilistycznych. Paul Erdős jako pierwszy w 1947 roku zaprezentował dowód z użyciem metod probabilistycznych na granicę dolną dla liczb R(k,k)\cite{erdogranica, graniceupdown}. Metoda ta opierała się na wykazaniu że w losowo pokolorowanym grafie $\mathit{K}_{n}$ prawdopodobieństwo znalezienia jednokolorowego grafu $\mathit{K}_{k}$ jest mniejsza od 1 dla pewnej wartości.

\end{enumerate}

\section{Artykuł R(4,5)=25}


Głównym źródłem wymagań naszej pracy jest publikacja R(4,5) = 25 wydana w 1995 przez Brendana D. McKaya oraz Stanisława P. Radziszowskiego  \cite{mainpaper} . Motywacją do sporządzenia tej pracy było udowodnienie ze dokładna wartość liczby Ramseya R(4,5) wynosi 25, z pomocą użycia nowych technologii. Prace nad wyznaczeniem wartości R(4,5) zaczęły się w 1955 wraz z wydaniem artykułu przez Greenwooda oraz Gleasona w którym wyznaczyli górną granicę R(4,5) $\leq$ 31. W kolejnych latach granica ta była zawężana aż do 25 $\leq$ R(4,5)  26.\par
Jak pokazano wcześniej, wygenerowanie wszystkich możliwych dwukolorowych grafów a następne ich sprawdzenie było by zbyt czasochłonne więc wymagane było inne podejście do problemu. Wykorzystano jedynie wyselekcjonowane grafy (s,t,n) gdzie s oznacza rozmiar kliki która znajduje się w grafie, t oznacza wielkość zbiory niezależnego który należy do grafu, a n oznacza liczbę wierzchołków na których zbudowany jest graf. Celem było skonstruowanie rodziny grafów (4,5,24) z grafów (3,5,d) oraz (4,4,24-d) gdzie 7 $\leq$ d $\leq$ 13, przy pomocy algorytmu nazwanym ''sklejaniem''. Ostatnim krokiem przed weryfikacją wyników jest próba rozszerzenia otrzymanych grafów (4,5,24) o jeden wierzchołek.

  
\section{Generowanie grafów}

\subsection{Generowanie grafów nieizomorficznych} 

\begin{algorithm}
  \caption{Generowanie grafów nieizomorficznych}
  \begin{algorithmic}
  \REQUIRE $n > 0 $
  \STATE grafy[] <- graf jednowierzchołkowy 
  \WHILE{$ilośćWierzchołków(grafy) < n$}
    \STATE dodajWierzchołek(grafy)
    \STATE usunIzomorfizmy(grafy)
  \ENDWHILE
  \end{algorithmic}
\end{algorithm}

W uproszczeniu: generowanie grafów rozpoczynamy od pojedynczego wierzchołka.
Następnie powtarzamy następujące kroki dopóki nie osiągniemy docelowej liczby wierzchołków:
\begin{enumerate}
 \item Generowanie grafów nieizomorficznych 

 \begin{itemize}
 \item  Dodanie nowego wierzchołka  

 \item Utworzenie wszystkich możliwych zestawów krawędzi zawierających ten wierzchołek. Przy dodawaniu n-tego wierzchołka mamy 

 \item Sprawdzenie izomorfizmów - jeżeli wygenerowany graf jest izomorficzny z innym wygenerowanym grafem, to pozostawiamy tylko jeden z nich 
 
 \end{itemize}
\end{enumerate}

Zaczynamy od pojedynczego wierzchołka.

\begin{figure}[H]
  \centering
  \begin{tikzpicture}[node distance={15mm}, main/.style = {draw, circle}] 
    \node[main] (1) {$1$};
   \end{tikzpicture}
   \caption{}
\end{figure}

\begin{figure}[H]
  \centering
  \begin{tikzpicture}[node distance={15mm}, main/.style = {draw, circle}] 
    \node[main] (2) [below of=1] {$1$};
    \node[main] (3) [right of=2] {$2$};
   
    \node[main] (4) [right of=3] {$1$};
    \node[main] (5) [right of=4] {$2$};
   
    \draw (2) -- (3);
   \end{tikzpicture}
   \caption{}
\end{figure}

W pierwszym kroku dodajemy drugi wierzchołek i otrzymujemy dwa różne grafy - jeden, w którym wierzchołki sąsiadują, drugi, w którym nie. 

\begin{figure}[H]
  \centering
  \begin{tikzpicture}[node distance={15mm}, main/.style = {draw, circle}] 
    \node[main] (1) {$1$};
    \node[main] (2) [right of=1] {$2$};
    \node[main] (3) [below of=1] {$3$};
   
    \draw (1) -- (2);
   \end{tikzpicture} 
   \hfill
   \begin{tikzpicture}[node distance={15mm}, main/.style = {draw, circle}] 
    \node[main] (1) {$1$};
    \node[main] (2) [right of=1] {$2$};
    \node[main] (3) [below of=1] {$3$};
   
    \draw (1) -- (2);
    \draw (1) -- (3);
   \end{tikzpicture} 
   \hfill
   \begin{tikzpicture}[node distance={15mm}, main/.style = {draw, circle}] 
    \node[main] (1) {$1$};
    \node[main] (2) [right of=1] {$2$};
    \node[main] (3) [below of=1] {$3$};
   
    \draw (1) -- (2);
    \draw (1) -- (3);
    \draw (2) -- (3);
   \end{tikzpicture} 
  
  \caption{Grafy wygenerowane z pierwszego grafu z poprzedniego kroku}
\end{figure}


Następnie bierzemy pierwszy z grafów i dodajemy do niego kolejny wierzchołek, ponownie generując wszystkie możliwości.  

\begin{figure}[H]
  \centering
  \begin{tikzpicture}[node distance={15mm}, main/.style = {draw, circle}] 
  \node[main] (1) {$1$};
  \node[main] (2) [right of=1] {$2$};
  \node[main] (3) [below of=1] {$3$};

  \end{tikzpicture}
  \hfill
  \begin{tikzpicture}[node distance={15mm}, main/.style = {draw, circle}] 
  \node[main] (1) {$1$};
  \node[main] (2) [right of=1] {$2$};
  \node[main] (3) [below of=1] {$3$};

  \draw (3) -- (2);
  \end{tikzpicture}
  \hfill
  \begin{tikzpicture}[node distance={15mm}, main/.style = {draw, circle}] 
  \node[main] (1) {$1$};
  \node[main] (2) [right of=1] {$2$};
  \node[main] (3) [below of=1] {$3$};

  \draw (3) -- (2);
  \draw (1) -- (3);
  \end{tikzpicture}
  \caption{{Grafy wygenerowane z drugiego grafu z poprzedniego kroku}}
\end{figure}
To samo powtarzamy dla grafu drugiego. 

Jak łatwo zauważyć graf 1 i graf 5 oraz graf 2 i graf 6 są parami izomorficzne, więc możne odrzucić po jednym z każdej pary, zmniejszając liczbę grafów rozważanych w kolejnych krokach.  


\begin{figure}[H]
  \centering
  \begin{tikzpicture}[node distance={15mm}, main/.style = {draw, circle}] 
    \node[main] (1) {$1$};
    \node[main] (2) [right of=1] {$2$};
    \node[main] (3) [below of=1] {$3$};
   
    \draw (1) -- (2);
   \end{tikzpicture} 
   \hfill
   \begin{tikzpicture}[node distance={15mm}, main/.style = {draw, circle}] 
    \node[main] (1) {$1$};
    \node[main] (2) [right of=1] {$2$};
    \node[main] (3) [below of=1] {$3$};
   
    \draw (1) -- (2);
    \draw (1) -- (3);
   \end{tikzpicture} 
   \hfill
   \begin{tikzpicture}[node distance={15mm}, main/.style = {draw, circle}] 
    \node[main] (1) {$1$};
    \node[main] (2) [right of=1] {$2$};
    \node[main] (3) [below of=1] {$3$};
   
    \draw (1) -- (2);
    \draw (1) -- (3);
    \draw (2) -- (3);
   \end{tikzpicture} 
   \hfill
  \begin{tikzpicture}[node distance={15mm}, main/.style = {draw, circle}] 
  \node[main] (1) {$1$};
  \node[main] (2) [right of=1] {$2$};
  \node[main] (3) [below of=1] {$3$};

  \end{tikzpicture}
  \caption{Wszystkie nieizomorficzne grafy 3-wierzchołkowe}
\end{figure}
=======
W celu uniknięcia porównywania wszystkich grafów, do nowo dodanego grafu wyznaczane są orbity - tylko jeden sposób generacji każdego grafu spowoduje sytuację, gdzie nowo dodany wierzchołek znajduje się w pierwszej orbicie, co pozwala łatwo odrzucać izomorfizmy.


   \subsection{Generowanie grafów Ramseya}
   Generowanie grafów obarczonych ograniczeniami co do rozmiarów klik oraz zbiorów niezależnych wygląda niemal tak samo jak generowanie grafów dowolnych, 
   ale zawiera dodatkowy krok - sprawdzenie czy graf nie zawiera kliki lub zbioru niezależnego stopnia który zaburzyłby jego ramseyowskość. Takie grafy można odrzucać, ze względu na następujące twierdzenie:
     \begin{theorem}
      Jeżeli graf G posiada klikę stopnia n, oraz istnieje graf H taki że G jest podgrafem H, to H również posiada klikę stopnia n.
   \end{theorem}
  Dzięki powyższemu twierdzeniu wiemy, że wszelkie próby rozszerzenia grafu posiadającego klikę stopnia n powodują powstanie grafów z przynajmniej tak wielką kliką. Analogiczne rozumowanie można przeprowadzić dla rozszerzania grafów ze zbiorem niezależnym o wystarczającej wielkości. Dzięki temu odrzucanie grafów nieramseyowskich na wczesnym etapie generacji nie powoduje odrzucenia żadnych pożądanych grafów. Takie odrzucanie znacznie zmniejsza ilość grafów powstających na kolejnych etapach generacji, zwłaszcza dla grafów większego rzędu, bez czego wygenerowanie nawet grafów wymaganych do sklejania byłoby trudne. Nieizomorficznych grafów 17 wierzchołkowych jest ponad $10^{26}$.\cite{OEIS} Dzięki wczesnym wykluczaniu dla R(4,4) generowany jest tylko jeden taki graf. 
  
Odrzucanie grafów nieramseyowskich przy każdym rozszerzeniu pozwala wprowadzić dodatkową optymalizację. Jeżeli wiemy, że graf przed rozszerzeniem nie zawierał niedozwolonej kliki/zb. niezależnego, to mamy gwarancję, że jeżeli nowo rozszerzony graf je zawiera, to nowy wierzchołek musi wchodzić w ich skład. Dzięki temu można uzyskać dalsze przyspieszenie algorytmu generacji grafów.


\section{Sklejanie grafów}
Efektywne generowanie dużych grafów jest bardzo wymagające zarówno czasowo jak i pamięciowo. Dlatego stosujemy technikę nazywaną sklejaniem grafów. 

\subsection{Idea sklejania}
Zbiór danych wymagany do poprawnego sklejania jest uzyskany poprzez wygenerowanie wszystkich nieizomorficznych grafów G rzędów 7-13 spełniających R(3,5), oraz wszystkich nieizomorficznych grafów H rzędów 11-17 spełniających R(4, 4).

Grafy dzielone są na grupy pod względem ich stopnia. Celem sklejania jest uzyskanie grafów rzędu 24 poprzez połączenie grafów G oraz H, a więc suma rzędów grup grafów G oraz H branych pod uwagę musi być równa 24. Ze względu na R(3,5) = 14 największy rząd grupy grafów G to 13, a ze względu na R(4,4) = 18 grafy H są ograniczone do 17 wierzchołków. Oznacza to, że istnieje 6 par grup spełniających wymagania.

Potencjalne krawędzie między grafami G oraz H są grupowane w stożki.

\begin{definition}
Prawdopodobny stożek (ang. feasible cone) to podzbiór wszystkich wierzchołków grafu H, który nie tworzy kliki $K_3$.
\end{definition}

Dla przeciętnego grafu R(4, 4, 14) znajdziemy około 4000 takich prawdopodobnych stożków. Jeżeli do każdego z wierzchołków grafu G przypiszemy taki stożek, uzyskamy takie połączenie grafów G oraz H gdzie każdy wierzchołek grafu G jest sąsiadem wszystkich wierzchołków w przypisanym mu stożku. Główna część algorytmu sklejania polega na eliminacji wszystkich kombinacji stożków, które prowadzą do powstania kilki stopnia 4 lub zbiór niezależny stopnia 5. Każdy taki zbiór stożków jest poprawnym połączeniem grafów G oraz H. Taki algorytm sklejania pozwala uzyskać grafy R(4,5) o dowolnym stopniu (poprzez inne parowanie grup grafów G i H), ale na nasze potrzeby tworzymy jedynie grafy 24 wierzchołkowe.




\subsection{Algorytm sklejania}
Pierwszym etapem sklejania grafów jest stworzenie dla danego grafu H zbioru stożków przed wybraniem odpowiadającego mu grafu G. Można odrzucić wszystkie stożki które obejmują klikę 3 rzędu ponieważ taki stożek eliminuje dowolny graf wynikowy (połączenie kliki 3 rzędu do dowolnego wierzchołka powoduje wystąpienie kliki 4 rzędu). Podejście naiwne, tj. pojedyncze sprawdzenie wszystkich potencjalnie poprawnych stożków nie jest jednak opłacalne ze względu na fakt, że potencjalnych stożków dla grafu n wierzchołkowego jest $2^n$. W celu przyspieszenia obliczeń stożki grupowane są w przedziały. 

\begin{definition}
Przedziałem to zbiór stożków ograniczony przez stożek górny i stożek dolny, który zawiera wszystkie stożki spełniające zależność  $X: B \subseteq X \subseteq T$. Przedział zapisany jest jako $[B, T]$ gdzie B oraz T to stożki grafu H. 
\end{definition}


Znajdowanie prawdopodobnych stożków rozpoczynamy od pojedynczego przedziału gdzie $B = \emptyset$, a T obejmuje wszystkie wierzchołki H (taki interwał zawiera wszystkie możliwe stożki). W celu wyodrębnienia poprawnych stożków korzystamy z poniższych własności:
\begin{enumerate}
\item Dla wierzchołka w, który spełnia $w \notin B$ oraz $w\in T$ prawdą jest że $[B, T] = [B + \{ w\}, T] \cup [B , T - \{ w\}]$ oraz $[B +\{ w\}, T] \cap [B , T - \{ w\}] = \emptyset$

\item Jeżeli stożek B zawiera klikę stopnia 3, to wszystkie stożki w przedziale również ją zawierają. 

\item Jeżeli stożek T nie zawiera kliki stopnia 3, to żaden stożek w przedziale jej nie zawiera.
\end{enumerate}

Korzystając z własności 2 i 3 jako warunku odpowiednio odrzucenia lub zaakceptowania przedziału zaimplementowaliśmy algorytm rekurencyjny dzielący interwały wzdłuż wierzchołków należących do B oraz kliki 3, dzięki czemu uzyskujemy zbiór przedziałów dla grafu H użyty w następnym etapie algorytmu.
Dalsze odrzucanie stożków musi odbyć się już w kontekście grafu G.
  

\subsection{Zawężanie przedziałów - zasady A-D}
Zdefiniujmy 3 funkcje pomocnicze działające na podzbiorze grafu H oznaczonym jako X, generujące podzbiór grafu H. 
Funkcja H1 będzie wybierała wszystkich sąsiadów wierzchołków ze zbioru X.
H2 wybiera wszystkie wierzchołki, które nie sąsiadują z jednym lub większą ilością wierzchołków spoza X. 
H3 wybiera wszystkie wierzchołki, które nie sąsiadują z dwoma lub większą ilością wierzchołków spoza X.
\begin{itemize}
    
  \item   $H_1(X) = { w \in VH | vw \in EX \textrm{ dla jakiegoś } v \in X }$ 
  
  \item   $H_2(X) = { w \in VH | vw \notin EX \textrm{ dla jakiegoś } v \notin X}$
  
  \item   $H_3(X) = { w \in VH | { u, v, w } \textrm{ jest zbiorem niezależnym dla jakichś } v, u \notin X}$ 
\end{itemize}
Kolejny etap zawężania potencjalnych krawędzi pomiędzy grafami G oraz H odbywa się już w kontekście konkretnego grafu G. Każda możliwa kombinacja wygenerowanych interwałów jest  przydzielana do wierzchołków należących do G, dzięki czemu wzięte pod uwagę są wszystkie możliwe pomiędzy tymi grafami sposoby połączenia. Dla każdego potencjalnego zestawu interwałów łączących grafy G oraz H wykonywany jest poniższy zbiór reguł:
\begin{itemize}
  \item   A - stosowana do 2 wierzchołków u,v $\in G$ sąsiadujących ze sobą \linebreak
  Jeśli $B_u \cap B_v \cap H_1(B_u \cap B_v)$ nie jest zbiorem pustym to nie da się 
  poprawnie skleić tej pary grafów. \linebreak
  W innym wypadku z $T_u$ usuwamy $H_1(B_u \cap B_v) \cap B_v $  
  \item   B - stosowana do 2 wierzchołków u,v $\in G$ nie sąsiadujących ze sobą \linebreak
  Jeśli $H_3(T_u \cup T_v) \not\subseteq (T_u \cup T_v)$ to nie da się poprawnie skleić tej 
  pary grafów. \linebreak
  W innym wypadku $B_u$ rozszerzamy do $B_u \cup (H_3(T_u \cup T_v) - T_v)$
  \item   C - stosowana do 3 wierzchołków u,v,w $\in G$ tworzących zbiór niezależny 3 \linebreak
  Jeśli $H_2(T_u \cup T_v \cup T_w) \not\subseteq (T_u \cup T_v \cup T_w)$ 
  to nie da się poprawnie skleić tej pary grafów. \linebreak
  W innym wypadku $B_u$ rozszerzamy do $B_u \cup (H_2(T_u \cup T_v \cup T_w) - (T_v \cup T_w))$
  \item   D - stosowana do 4 wierzchołków u,v,w,z $\in G$ tworzących zbiór niezależny 4 \linebreak
  Jeśli $T_u \cup T_v \cup T_w \cup \neq VH $ to nie da się poprawnie skleić tej pary grafów.
  W innym wypadku $B_u$ rozszerzamy do $B_u \cup (VH - (T_v \cup T_w \cup T_z))$
\end{itemize}

Zasada A sprawdza czy sąsiednie wierzchołki u, v mają moją jakichkolwiek wspólnych potencjalnych sąsiadów, którzy są sąsiedni względem siebie.
W takiej sytuacji powstałaby klika $K_4$, więc nie uda się utworzyć grafu. 

Jeżeli tak nie jest, musimy usunąć z $T_u$ niezbędnych sąsiadów v, 
którzy sąsiadują z przynajmniej jednym z niezbędnych sąsiadów u. \linebreak

Zasada B sprawdza czy istnieją w grafie H zbiory niezależne $N_3$, z których wierzchołkami żaden z wierzchołków u, v 
nie może zostać połączony. W takim wypadku nie da się skleić grafów bez utworzenia zbioru niezależnego $N_5$. 

W innej sytuacji musimy do zbioru $B_u$ dodać wszystkie wierzchołki, 
z którymi przynajmniej 2 wierzchołki spoza zbioru $T_u \cup T_v$ nie sąsiadują. Wyjątkiem są te, z którymi można połączyć v.  
 
\paragraph{}
Zasada C sprawdza czy istnieją w grafie H dwa nie sąsiadujące wierzchołki, z którymi żaden z wierzchołków zbiory niezależnego u, v, w nie może zostać połączony. 
Oznaczałoby to, że musi powstać zbiór niezależny $N_5$, a grafów nie można skleić.


W innym wypadku do $B_u$ trzeba dodać wierzchołki które nie sąsiadują z wierzchołkiem, 
z którym żaden z wierzchołków u, v, w nie może zostać połączony, z wyjątkiem tych z którymi może sąsiadować v lub w.  \linebreak

Zasada D sprawdza czy graf H zawiera wierzchołek, z którym żaden z nie sąsiadujących wierzchołków u, v, w, z
nie może sąsiadować. Ponownie, w takim przypadku powstałby zbiór niezależny $N_5$.


Jeżeli powyższe nie zajdzie, musimy dodać do $B_u$ wszystkie wierzchołki, z którymi v, w, z nie mogą sąsiadować. 


\paragraph{}Warto zauważyć, że wynikiem zastosowania dowolnej reguły jest zawsze odrzucenie zestawu przedziałów, zawężenie przedziału lub brak zmian. Reguły są więc aplikowane do momentu w którym ponowne zaaplikowanie dowolnej reguły dla dowolnych wierzchołków nie powoduje już dalszych zmian w interwałach. Wynikowy zbiór interwałów daje sposób lub sposoby połączenia grafów G oraz H w graf 24 wierzchołkowy z zadowoleniem wymagań pod względem ramseyowskości.

Jeżeli powyższe nie zajdzie, musimy dodać do $B_u$ wszystkie wierzchołki, z którymi v, w, z nie mogą sąsiadować. \linebreak


Warto zauważyć, że wynikiem zastosowania dowolnej reguły jest zawsze odrzucenie zestawu przedziałów, zawężenie przedziału lub brak zmian. Reguły są więc aplikowane do momentu w którym ponowne zaaplikowanie dowolnej reguły dla dowolnych wierzchołków nie powoduje już dalszych zmian w interwałach. Wynikowy zbiór interwałów daje sposób lub sposoby połączenia grafów G oraz H w graf 24 wierzchołkowy z zadowoleniem wymagań pod względem ramseyowskości.


\subsection{Wynik sklejania}

Po zakończeniu procesu sklejania uzyskujemy zbiór grafów R(4,5,24). W dalszej części pracy graf z tego zbioru będzie oznaczany jako graf F. Jeżeli zbiór R(4,5,25) $\neq \emptyset$ to w wyniku rozszerzenia przynajmniej jednego z grafów F o jeden wierzchołek powinniśmy uzyskać przynajmniej jeden graf 25 wierzchołkowy spełniający R(4,5). 
Rozszerzanie o jeden wierzchołek zostaje w tym wypadku wykonane metodą podobną do wcześniejszego tworzenia przedziałów. Pierwszym etapem jest jednak stworzenie listy grup wierzchołków F które tworzą kliki lub zbiory niezależne. Celem tej listy jest przyspieszenie eliminacji rozszerzeń, które nie są poprawne.
Podobnie jak w wypadku sklejania tworzony jest przedział zbiorów wierzchołków opisany zbiorami: górnym i dolnym. Ma on te same własności, co uprzednie przedziały:

Dodatkowo, można jeszcze wyodrębnić dwie własności które nie miały znaczenia dla wcześniejszych przedziałów:
\begin{enumerate}
\item Jeżeli zbiór B zawiera przynajmniej jeden wierzchołek ze zb. niezależnego stopnia 5, to wszystkie zbiory w przedziale również go nie zawierają. 

\item Jeżeli zbiór T nie zawiera żadnego wierzchołka ze zb. niezależnego stopnia 5, to żaden stożek w przedziale ich nie zawiera.
\end {enumerate}
Dwie powyższe, wraz z wcześniejszymi własnościami pozwalają stworzyć algorym masowo eliminujący wierzchołki z przedziału na podstawie listy klik oraz zbiorów niezależnych, a w efekcie określenie czy istnieje rozszerzenie grafu zadowalające R(4,5).

\section{Implementacja i eksperymenty}
Nasz program został zaimplementowany w języku C z wykorzystaniem biblioteki Nauty, która umożliwia łatwe obliczeniowo wykrywanie orbit, co znacznie przyspiesza proces generacji grafów. 
Autor biblioteki jest profesor McKay. 

Dodatkowo, jednym z aspektów powyższego kodu, który zaadoptowaliśmy do własnego użytku jest sposób przechowywania grafów w pamięci komputerowej, który bazuje na macierzy sąsiedztwa. Macierz sąsiedztwa to sposób przechowania grafu o N wierzchołkach przy użyciu macierzy kwadratowej o wymiarach NxN. Wartość na pozycji m, n odpowiada istnieniu krawędzi pomiędzy wierzchołkami m oraz n w przypadku wartości 1, lub jej braku w przypadku wartości 0.
\begin{figure}[h]
\begin{subfigure}{0.5\textwidth}
\begin{tikzpicture}[node distance={15mm}, main/.style = {draw, circle}] 
    \node[main] (1) {$1$};
    \node[main] (2) [below right of=1] {$5$};
    \node[main] (3) [above right of=2] {$2$};
    \node[main] (4) [below left of=2] {$4$};
    \node[main] (5) [below right of=2] {$3$};

    \draw (1) -- (2);
    \draw (2) -- (3);

    \draw (2) -- (5);
    \draw (1) -- (3);

	

  \end{tikzpicture}
  \end{subfigure}
  \begin{subfigure}{0.5\textwidth}
$\begin{Bmatrix}
	0 & 1 & 0 & 0 & 1 \\
	1 & 0 & 0 & 0 & 1 \\
	0 & 0 & 0 & 0 & 1 \\
	0 & 0 & 0 & 0 & 0 \\
	1 & 1 & 1 & 0 & 0 \\
	\end{Bmatrix}$
	
	\end{subfigure}
	\caption{Graf wraz z odpowiadającą mu macierzą sąsiedztwa}
\end{figure}

Warto zauważyć nadmiarowość macierzy, gdzie każdej krawędzi odpowiadają dwie wartości 1. Ta nadmiarowość okazjonalnie pozwala na przyspieszenie obliczeń w zmodyfikowanej wersji używanej w naszym kodzie.
Moedyfikacja metody macierzy sąsiedztwa zastosowane w naszym kodzie polega na odejściu od zapisywania każdej liczby w macierzy jako osobnej wartości. Jako że zajmujemy się jedynie grafami prostymi, to wartości w poszczególnych komórkach mogą wynosić jedynie 0 lub 1. W związku z tym wiersz macierzy można zapisać nie jako n wartości, a jako jedną wartość o odpowiedniej ilości bitów. Ze względu na to, że największym grafem występującym w naszej pracy jest potencjalny graf 25 wierzchołkowy, 32 bitowa wartość jest wystarczająca żeby pomieścić dowolny wiersz macierzy występujący w naszym kodzie. 
 

\begin{thebibliography}{9}
  \bibitem{mainpaper} 
  Brendan D. McKay, Stanisław P. Radziszowski. 
  \textit{R(4,5) = 25}. 
  
  \bibitem{nauty} 
  Brendan D. McKay.
  \textit{nauty user's guide (version 2.4)}.  
  Department of Computer Science
  Australian National University
  Canberra ACT 0200, Australia
  November 4, 2009

  \bibitem{smallramsey} 
  Stanisław P. Radziszowski.
  \textit{Small Ramsey Numbers}.  
  Department of Computer Science Rochester Institute of Technology Rochester, NY 14623
  June 11, 1994

  \bibitem{cykl4} 
  Janusz Dybizbański.
  \textit{Liczby Ramseya z cyklem C4}.  
  Uniwersytet Warszawski
  Wydział Matematyki, Informatyki i Mechaniki
  Listopad 2013

  \bibitem{numbers} 
  Christos Nestor Chachamis.
  \textit{Ramsey Numbers}.  
  May 13, 2018

  \bibitem{theory} 
  Lane Barton IV
  \textit{Ramsey Theory}.  
  May 13, 2016

  \bibitem{generation} 
  Brendan D. McKay
  \textit{ISOMORPH-FREE EXHAUSTIVE GENERATION}.  
  May 13, 2016

  \bibitem{erdogranica} 
  P.  Erdősa,  Some  Remarks  on  the  Theory  of  Graphs,
  \textit{Bulletin  of  the  American  Mathematical  Society},
  53(4):292–294,  1947
  

  \bibitem{graniceupdown} 

  Christos Nestor Chachamis
  \textit{Ramsey Numbers}.  
  May 13, 2018

 \bibitem{ramsey} 
 Ramsey, F.P. (1930), 
 \textit{On a Problem of Formal Logic.}
 Proceedings of the London Mathematical Society, s2-30: 264-286. https://doi.org/10.1112/plms/s2-30.1.264
 
 \bibitem{erdoshappy} 
 Erdős, P.; Szekeres, G.
 A combinatorial problem in geometry. Compositio Mathematica, Volume 2 (1935) , pp. 463-470.
  
   \bibitem{aliens} 
  \textit{Ramsey Theory}
   by Ronald L. Graham and Joel H. Spencer, in Scientific American (July 1990), p. 112-117
   
   \bibitem{OEIS}
       OEIS Foundation Inc. (2020), The On-Line Encyclopedia of Integer Sequences, http://oeis.org/A000088.

  \end{thebibliography}

\end{document}
